%!TEX root = ../../Paper.tex

\chapter{Introduction}
\label{cha:introduction}

\section{Motivation}
\label{sec:motivation}
The immense rise of social media is one of the driving forces behind the current Big Data trend. Big data creates 2.5 billion gigabytes every day and produced 90\% of the worldwide data in the last two years, thus it has become a top priority for research organizations and companies \cite{ibm2012bigdata}. The combination of Big Data and powerful analytical technologies makes it possible to gain highly valuable insights that otherwise might not be accessible. 

The popularity of social media services, including social networks, micro-blogging tools, wikis, and photo and video-sharing applications has increased exponentially in the last few years \cite{cambria2013big}. Social media allows individuals and organizations to capture and understand the imaginations, opinions, ideas, conversations and feelings of millions of people. As social media services continue to proliferate, the amount of unstructured social data keeps growing. 

\todo{Rewrite this part}Emerging Big Data and advanced Natural Language Processing technologies make it possible to collect and analyze those massive amounts of data and enables a fundamentally new approach for the study of society and human beings. 

When hurricane Sandy hit the US Eastcoast on October 29 2012, government agencies and individuals turned to social media services \enquote{to communicate with the public like never before} \cite{emergencymgmt2013sandy}. Hurricane Sandy \enquote{marked a shift in the use of social media in disasters} \cite[6]{security13sandy} and attracted many data researchers to monitor and analyze this event \cite{kumar2011tweettracker,carageamapping2014sandy}. Besides the analysis of natural disasters, big social data analysis has been shown to be useful for many other use cases: The FBI utilizes advanced data analytic technologies to predict crimes and terrorist attacks based on publicly available social data \cite{wang2012automatic}. Several research projects leveraged those technologies on big social data to predict the spread of diseases \cite{Google09detection,gft2014}. Moreover, social media analysis has been proven to predict political sentiment and forecast election winners \cite{bermingham2011using}. These successful results of mainly research-based projects helped to open up new business opportunities. Companies already use social media monitoring and analysis techniques to predict the stock market in real time \cite{bollen2011twitter,alcorn2013stockmarket}. Further, an increasing number of companies utilize these technologies to analyze the customer satisfaction and research the public opinion about products and their company itself \cite{journals/expert/CambriaSXH13}. In addition, newspaper publishers use big social data analysis to mine the public interest and predict how popular their stories might become.

Big social data analysis has grown into a serious business over the past several years and nowadays includes disciplines such as social media analytics, sentiment analysis, social network analysis, trend discovery and opinion mining.


\section{Objectives}
\label{sec:objectives}
In the beginning of the project, we wanted to analyze Stack Overflow. Stack Overflow is one of the biggest Q\&A pages of the today’s web and the flagship of the Stack Exchange Network.
Our goal was to get high-quality insights into trending topics of developers around the globe. 
After identifying current hot topics people write about, we wanted to search Twitter messages for the same topics. As a result, we wanted to find out if it is possible to discover trends we identified on Stack Overflow also on Twitter. In the next step, we wanted to categorize and analyze detected intersection on both media platforms. The project was supposed to answer among other possible questions the following ones: 
Is Twitter used to ask questions?
Is there a chronological difference between the uprising of a trend on Stack Overflow and Twitter?
Are there opinion leaders in one of the sources? [People who ask a lot of questions / tweet a lot about a topic]

After a renewed validation of the project’s purpose we shifted the direction. We had the assumption that we would find only a few intersections between topics discussed on Stack Overflow and Twitter, if any. Additionally, Stack Overflow already offers quite sophisticated statistics about its data, including topics. These statistics make an own analysis redundant.

As a consequence, we changed the project’s objective, which is depicted in the following.
[Check and adapt the following paragraph depending on the real content of our project]
The goal of the project is the early detection and prediction of arising trends on Twitter.
We assume that it is possible to predict the spreading of future trends on Twitter based on the curves of trends in the past. 
Therefor, we want to explore different metrics and dimensions, such as retweets, hashtag/topic occurrences, user groups and emotions.
It helps to detect big headlines before they go viral and, therefore, it is very valuable in different areas such as stock market, brand awareness, political discussions and elections and the success of media (movies, music). 

We suggest an architecture consisting of two systems for data collection. 
The first system is used to monitor the entire Twitter stream and focused on detecting on trends that are in early stage. Furthermore, it uses topic modeling to identify topics/hashtags that are correlated to the same trend.
These results are then forwarded to the second system. The second system utilizes this data for observing only those topics in detail until they are not relevant anymore. 

In the next step, we plan to use (unsupervised) machine learning techniques to compare the early trends with previous trend curves to predict their further course.

Additionally, we may compare the overall results with data from Google Trends to check for similarities.


\section{Overview}
\label{sec:overview}
This paper is structured into five chapters. The remaining part of the paper is organized as following.

\textbf{Chapter~\ref{cha:background}} provides the theoretical background of this paper. Used terms such as Big Data, Social Media and several data processing techniques are described. 
\todo{Probably leave Use Case chapter out, because we have the user stories chapter}
The architecture to collect Twitter data and different analyzing techniques are depicted in \textbf{chapter~\ref{cha:trend-detection-concept}}. Furthermore, we mention some works that are related to ours.
In \textbf{chapter~\ref{cha:trend-stories}} we present three detected trends. Additionally, we present the results of the utilized techniques and present interpretations of the data.
\textbf{Chapter~\ref{cha:conclusion-future-work}} provides a conclusion and an overview about possible future work.