%!TEX root = ../../Paper.tex

\chapter{Introduction}
\label{cha:introduction}

\section{Motivation}
\label{sec:motivation}
The immense rise of social media is one of the driving forces behind the current Big Data trend. Big data creates 2.5 billion gigabytes every day and produced 90\% of the worldwide data in the last two years, thus it has become a top priority for research organizations and companies \cite{ibm2012bigdata}. The combination of Big Data and powerful analytical technologies makes it possible to gain highly valuable insights that otherwise might not be accessible. 

The popularity of social media services, including social networks, micro-blogging tools, wikis, and photo and video-sharing applications has increased exponentially in the last few years \cite{cambria2013big}. Social media allows individuals and organizations to capture and understand the imaginations, opinions, ideas, conversations and feelings of millions of people. As social media services continue to proliferate, the amount of unstructured social data keeps growing. 

\todo{Rewrite this part}Emerging Big Data and advanced Natural Language Processing technologies make it possible to collect and analyze those massive amounts of data and enables a fundamentally new approach for the study of society and human beings. 

When hurricane Sandy hit the US Eastcoast on October 29 2012, government agencies and individuals turned to social media services "to communicate with the public like never before" \cite{emergencymgmt2013sandy}. Hurricane Sandy "marked a shift in the use of social media in disasters" \cite[6]{security13sandy} and attracted many data researchers to monitor and analyze this event \cite{kumar2011tweettracker,carageamapping2014sandy}. Besides the analysis of natural disasters, big social data analysis has been shown to be useful for many other use cases: The FBI utilizes advanced data analytic technologies to predict crimes and terrorist attacks based on publicly available social data \cite{wang2012automatic}. Several research projects leveraged those technologies on big social data to predict the spread of diseases \cite{Google09detection,gft2014}. Moreover, social media analysis has been proven to predict political sentiment and forecast election winners \cite{bermingham2011using}. These successful results of mainly research-based projects helped to open up new business opportunities. Companies already use social media monitoring and analysis techniques to predict the stock market in real time \cite{bollen2011twitter,alcorn2013stockmarket}. Further, an increasing number of companies utilize these technologies to analyze the customer satisfaction and research the public opinion about products and their company itself \cite{journals/expert/CambriaSXH13}. In addition, newspaper publishers use big social data analysis to mine the public interest and predict how popular their stories might become.

Big social data analysis has grown into a serious business over the past several years and nowadays includes disciplines such as social media analytics, sentiment analysis, social network analysis, trend discovery and opinion mining.


\section{Objectives}
\label{sec:objectives}


\section{Overview}
\label{sec:overview}