%!TEX root = ../../Paper.tex

\chapter{Introduction}
\label{cha:introduction}

\section{Motivation}
\label{sec:motivation}
The immense rise of social media is one of the driving forces behind the current Big Data trend. With 2.5 billion gigabytes of data created every day and 90\% of the worldwide data created in the last two years \cite{ibm2012bigdata}, Big Data has become a top priority for research organizations and companies. The combination of Big Data and powerful analytical technologies makes it possible to gain highly valuable insights that otherwise might not be accessible.

The popularity of social media services, including social networks, micro-blogging tools, wikis, and photo and video-sharing applications has increased exponentially in the last few years \cite{cambria2013big}. Social media allows individuals and organizations to capture and understand the thoughts, opinions, ideas, conversations and feelings of millions of people. As social media services continue to proliferate, the amount of unstructured social data keeps growing.

Emerging Big Data Analysis and advanced Natural Language Processing technologies make it possible to collect and analyze a massive amount of data and enables a fundamentally new approach for the study of society and human beings.

When hurricane Sandy hit the US Eastcoast on October 29 2012, government agencies and individuals turned to social media services \enquote{to communicate with the public like never before} \cite{emergencymgmt2013sandy}. Hurricane Sandy \enquote{marked a shift in the use of social media in disasters} \cite[6]{security13sandy} and attracted many data researchers to monitor and analyze this event \cite{kumar2011tweettracker,carageamapping2014sandy}. Besides the analysis of natural disasters, big social data analysis has been shown to be useful for many other use cases: The FBI utilizes advanced data analytic technologies to predict crimes and terrorist attacks based on publicly available social data \cite{wang2012automatic}. Several research projects leveraged those technologies on big social data to predict the spread of diseases \cite{Google09detection,gft2014}. Moreover, social media analysis has been proven to predict political sentiment and forecast election winners \cite{bermingham2011using}. These successful results of mainly research-based projects helped to open up new business opportunities. Companies already use social media monitoring and analysis techniques to predict the stock market in real time \cite{bollen2011twitter,alcorn2013stockmarket}. Further, an increasing number of companies utilize these technologies to analyze the customer satisfaction and research the public opinion about products and their company itself \cite{journals/expert/CambriaSXH13}. In addition, newspaper publishers use big social data analysis to mine the public interest and predict how popular their stories might become. In general, the detection, analysis and prediction of trends in an early stage makes it possible to identify big headlines before they go viral. Therefore, detection is very valuable in different areas such as stock market, brand awareness, political discussions and elections and the success of media. 
Big social data analysis has grown into a serious business over the past several years and nowadays includes disciplines such as social media analytics, sentiment analysis, social network analysis, trend discovery and opinion mining.


\section{Objectives}
\label{sec:objectives}
The goal of this project is the early detection and analysis of arising trends on Twitter. The desired outcome is a concept of a data pipeline and a full implementation of several components to collect, analyze and visualize Twitter data. In addition, this system shall be able to detect trends in an early stage to predict the spreading of future trends on Twitter based on the curves and features of trends in the past. Therefore, we want to explore different metrics and dimensions, such as retweets, hashtag and user mention occurrences, user groups and emotions. Furthermore, this paper explains various related implementations, recommends appropriate technologies and common methods to conduct a big social data analysis on Twitter trends. Our implementation will be presented based on the case study of Twitter data related to several major events happened between 15\textsuperscript{th} of December 2014 and 15\textsuperscript{th} of January 2015.
Additionally, we compare our results with data from Google Trends to check for similarities and the validity of detecting trends on Twitter based on a small statistical sample.

\clearpage
\section{Overview}
\label{sec:overview}
This paper is structured into five chapters. The remaining part of the paper is organized as following.

\textbf{Chapter~\ref{cha:background}} provides the theoretical background of this paper. Used terms such as Big Data, Social Media and several data processing techniques are described in this chapter.

The architecture to collect Twitter data and several techniques to analyze and visualize those data are presented in \textbf{chapter~\ref{cha:trend-detection-concept}}. Furthermore, we mention some related research and technologies.

In \textbf{chapter~\ref{cha:trend-stories}} we analyze three detected Twitter trends with several analysis techniques and explain the gained insights from them.

\textbf{Chapter~\ref{cha:conclusion-future-work}} provides a conclusion and an overview about possible future work.
