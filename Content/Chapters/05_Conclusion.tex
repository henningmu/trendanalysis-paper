%!TEX root = ../../Paper.tex

\chapter{Conclusion and Future Work}
\label{cha:conclusion-future-work}
Big social data analysis has grown into a serious business over the past several years with important use cases not just for research projects, but also in commercial products. Social Data analysis techniques are applied to predict terrorist attacks, stock performance, election results or the spread of diseases. Further, it is utilized by companies to analyze their customer’s satisfaction and the public opinion about their products. Cutting edge machine-learning, NLP and data mining technologies are necessary to gain valuable insights into large amounts of social content.

In this paper, we presented our concept and implementation of several components to collect, analyze and visualize Twitter data. Thereby, we collected about 18 million English tweets from the USA. Based on this dataset, we built a component to detect the spreading of trends at an early stage based on occurrences of hashtags and user mentions. With this system, we were able to detect several major events at an early stage such as the Air Asia tragedy and the PSN Hack, only based on the usage frequency of hashtags and mentions from a statistical sample of the Twitter dataset. The early detection can be further improved by analyzing all words of a tweet, since hashtags mostly get created and used within a short time delay to the actual event and only a small number\footnote{Only about 13\% of tweets in our dataset included hashtags} of tweets even contain hashtags.

We used several machine-learning and data mining techniques to analyze the data, such as bag-of-words, sentiment analysis, topic modeling and data classification. Finally, we visualized the trends with various visualization techniques. The development of a trend and the characteristic of events were analyzed by using time series visualizations. Further, we used geospatial visualizations to show the regional activity and emphasize different aspects about the local spreading of trends. Moreover, word clouds were used to highlight commonly used terms in the detected trend.

We conclude that detecting and analyzing Twitter for trends provides high-quality insights into the worlds interest in global events. Twitter trends even resemble regional interests in certain topics with a high accuracy. Further, Twitter is one of the fastest ways to detect and predict global trends, due to its open access to real-time data. However, to get more overall accurate results, it would be necessary to eliminate the current limitations by collecting a larger portion of the Twitter stream and offering multi-language support.
